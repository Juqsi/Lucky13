%! Author = Justus Benedict Siegert
%! Date = 22.01.2023

% Preamble
\documentclass[a4paper,11pt]{scrartcl}

\bibliographystyle{plain}

% Packages
\usepackage[utf8]{inputenc}
\usepackage[ngerman]{babel}
\usepackage[T1]{fontenc}


% Document
\begin{document}
    \title{Lucky Thirteen}
    \author{Justus Benedict Siegert}
    \date{8445024, s222312@student.dhbw-mannheim.de}
    \maketitle
    \tableofcontents


    \section{Einführung}\label{sec:einfuhrung}
    Lucky 13 ist eine Zeitattacke gegen das Transport Layer Security (TLS) und dem Datagram Transport Layer (DTLS),
    die 2013 von den Wissenschaftlern Nadhem J. AlFardan and Kenneth G. Paterson entdeckt wurde.
    Sie arbeiten bei der Royal Holloway University of London


    \section{(D)TLS}\label{sec:(d)tls}


    \section{Voraussetzungen für einen Angriff}\label{sec:voraussetzungen-fur-einen-angriff}
    Der Angriff wurde nur getestet, bei dem der Angreifer sich im selben Netzwerk befunden hat wie der Server\cite[S. 11]{AlFardan2013}


    \section{Angriffsverlauf}\label{sec:angriffsverlauf}


    \section{Sicherheitsbewertung}\label{sec:sicherheitsbewertung}


    \section{Gegenmaßnahmen}\label{sec:gegenmanahmen}
    Es gibt Grundsätzlich 3 Maßnahmen die ergriffen werden können um eine solche Angriffswahrscheinlichkeit zu minimieren

    \subsection{Zufällige Zeitverzögerungen}

    \subsection{RC4-Chiffre}

    \subsection{Implementierung der MEE-TLS-CBC-Entschlüsselung}
    \newpage

    \bibliography{Literatur/Quellen}
\end{document}

