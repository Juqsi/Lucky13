%! Author = Justus Benedict Siegert
%! Date = 22.01.2023

% Preamble
\documentclass[a4paper,11pt]{scrartcl}

\bibliographystyle{plain}

% Packages
\usepackage[utf8]{inputenc}
\usepackage[ngerman]{babel}
\usepackage[T1]{fontenc}


% Document
\begin{document}
    \title{Lucky Thirteen}
    \author{Justus Benedict Siegert}
    \date{8445024, s222312@student.dhbw-mannheim.de}
    \maketitle
    \tableofcontents
    \newpage


    \section{Einführung}\label{sec:einfuhrung}
    Das Verschlüsseln von Daten ist seit der Nutzung des Internets ein wichtiges Anliegen der Nutzenden.
    Das am häufigsten genutzte Protokoll ist das Transport Layer Security Protokoll (TLS).
    Lucky 13 ist eine kryptografische Schwachstelle des TLS-Protokolls in der Version 1.1, sowie dem Vorgänger TLS 1.0, SSL, die Maßnahmen gegen padding oracle attacks enthalten und dem Datagram Transport Layer Security (DTLS)~\cite[S.2]{AlFardan2013},
    bei dem die Daten kompromittiert werden.
    Durch die Schwachstelle können Angreifer Daten ausspionieren, die verschlüsselt wäre und vermeintlich sicher sind.
    Die Sicherheitslücke wurde 2013 von den Wissenschaftlern Nadhem J. AlFardan and Kenneth G. Paterson, der Security Group der Royal Holloway University of London entdeckt.
    \newline
    Der Lucky 13 Angriff ist eine Art von Man in the Middle attack verbunden mit einer padding oracle attack.
    Dabei ist das Problem, dass das Padding nach der Berechnung des message authentication code (MAC) hinzugefügt wird und somit unauthentifizierte Daten im verschlüsselten Klartext bildet.

    \subsection{Man in the middle attack}\label{subsec:man-in-the-middle-attack}
    Eine man in the middle Attacke ist ein Angriff, bei dem eine dritte Partei eine vermeintlich direkte Verbindung von zwei anderen zwischen schaltet.

    \subsection{padding oracle attack}\label{subsec:padding-oracle-attack}
    Bei einem Padding oracle attack wird ausgenutzt, dass chiffren für eine Blockverschlüsslung mit einem Padding aufgefüllt werden.
    Damit der Angriff aber auch gelingt muss der Server eine Antwort geben, ob das hinzugefügt Padding korrekt oder nicht ist.
    Durch die Antworten des Servers, welche genutzt werden können, um die Verschlüsslung zu knacken, kommt der name Orakel


    \section{(Datagram) Transport Layer Security}\label{sec:(d)tls}
    Der vorgänger von TLS, Secure Socket Layer (SSL) 1.0 erschien im Jahr 1994, neun Monate nach der ersten Mosaic version.
    In den folgenden zwei Jahre wurden zwei weiter Versionen entwickelt und dann SSL in TLS 1.0 umbenannt.
    \newline
    TLS besteht aus zwei Hauptkomponenten: dem TLS Handshake, der für den sicheren Schlüsselaustausch zwischen dem Client und dem Server zuständig ist,
    und dem TLS Record, dieser verwendet die beim Handshake ausgehandelten Schlüssel, um eine sichere Datenübertragung zu ermöglichen.
    Die Daten werden verschlüsselt und mit einem MAC gegen manipulation geschützt.
    Die grundlegende Funktionsweise besteht darin, dass der Client eine Verbindung mit dem Server aufbaut, welcher sich mit einem Zertifikat authentifiziert.
    Der Client überprüft die Vertrauenswürdigkeit und vergleicht die Daten des Zertifikats, mit den des Servers.
    \newline DTLS basiert auf TLS, kann aber im Gegensatz zu TLS auch Transportprotokolle wie UDP übertragen, hat aber grundlegend dieselben Eigenschaften und Sicherheitsmerkmale

    \subsection{CBC-Mode}\label{subsec:cbc-mode}
    CBC steht für Cipher Block Chaining Mode, ist eine Betriebsart in der Blockchiffre, dabei wird vor dem Verschlüsseln des Klartextblocks dieser mit dem vorausgehenden verschlüsselten Block per XOR verknüpft.
    Die Vorteiler dieses Modus sind, dass gleich Klartextblöcke unterschiedliche verschlüsselte Blöcke ergibt.


    \section{Voraussetzungen für einen Angriff}\label{sec:voraussetzungen-fur-einen-angriff}
    Damit der Angriff gelingt, muss die TLS version 1.1 oder eine ältere version mit Maßnahmen gegen padding oracle attacks.
    Bei anderen älteren Versionen gelingen sonst andere Angriffe.
    Der Angriff wurde nur getestet, bei dem der Angreifer sich im selben Netzwerk befunden hat wie der Server\cite[S. 11]{AlFardan2013}


    \section{Angriffsverlauf}\label{sec:angriffsverlauf}


    \section{Sicherheitsbewertung}\label{sec:sicherheitsbewertung}


    \section{Gegenmaßnahmen}\label{sec:gegenmanahmen}
    Es gibt mehrer Maßnahmen die ergriffen werden können um einen Angriffserfolg zu minimieren.

    \subsection{Zufällige Zeitverzögerungen}

    \subsection{RC4-Chiffre}

    \subsection{Implementierung der MEE-TLS-CBC-Entschlüsselung}
    \newpage

    \bibliography{Literatur/Quellen}
\end{document}

